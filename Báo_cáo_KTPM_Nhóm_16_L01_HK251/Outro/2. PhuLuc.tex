\section{Phụ lục}
\subsection{ADR 00: Lựa chọn kiến trúc ban đầu: Microservices Architecture (MSA)} \label{adr:00}

\begin{enumerate}
	\item \textbf{Tiêu đề:} Lựa chọn kiến trúc ban đầu: Microservices Architecture (MSA).
	\item \textbf{Status (Trạng thái):} Superseded (Đã bị thay thế).
	\item \textbf{Context (Bối cảnh):}
	\begin{enumerate}
		\item Hệ thống dự kiến sẽ phục vụ lượng lớn sinh viên, đặc biệt là vào các đợt cao điểm như mùa thi, đòi hỏi khả năng co giãn tối đa (Maximum Scalability).
		\item Cần tối ưu hóa hiệu năng cho các Module AI (\ref{nfr:18}, \ref{nfr:25}) bằng cách sử dụng các công nghệ khác nhau (Polyglot) cho từng thành phần.
		\item Các yêu cầu về \textbf{Elasticity} (\ref{nfr:23}, \ref{nfr:24}) và \textbf{Concurrency} (\ref{nfr:21}) là rất cao và cần được hỗ trợ bằng việc cô lập tài nguyên cho từng tính năng.
	\end{enumerate}
	\item \textbf{Decision (Quyết định):}
	\begin{enumerate}
		\item Áp dụng \textbf{Microservices Architecture (MSA)} để phân tách các Module chức năng thành các dịch vụ độc lập (ví dụ: Quiz Generation Service, User Service).
		\item Mỗi dịch vụ sẽ có khả năng \textbf{triển khai độc lập} và \textbf{mở rộng độc lập} (Autoscaling) dựa trên tải trọng CPU/GPU hoặc số lượng yêu cầu.
		\item Sử dụng \textbf{Eventual Consistency} cho các giao tiếp liên Service, chấp nhận độ trễ nhỏ để đạt được tính độc lập cao.
	\end{enumerate}
	\item \textbf{Consequences (Hệ quả):}
	\begin{enumerate}
		\item \textbf{Khả năng mở rộng tối đa (Maximum Elasticity):} Đáp ứng tốt yêu cầu \ref{nfr:25} bằng cách mở rộng các dịch vụ AI một cách chi tiết.
		\item \textbf{Tăng độ phức tạp về vận hành (DevOps):} Yêu cầu đội ngũ vận hành mạnh và các công cụ giám sát phức tạp.
		\item \textbf{Rủi ro về Data Consistency:} Phải quản lý giao dịch phân tán (Saga) giữa các CSDL riêng biệt, làm tăng độ phức tạp trong việc đảm bảo \ref{nfr:16}.
	\end{enumerate}
\end{enumerate}

\subsection{ADR 01: Chuyển đổi kiến trúc từ Microservices (MSA) sang Service-Based Architecture (SBA)} \label{adr:01}

\begin{enumerate}
	\item \textbf{Tiêu đề:} Chuyển đổi kiến trúc từ Microservices (MSA) sang Service-Based Architecture (SBA).
	\item \textbf{Status (Trạng thái):} Accepted (Đã chấp nhận).
	\item \textbf{Context (Bối cảnh):}
	\begin{enumerate}
		\item Hệ thống có yêu cầu cao về \textbf{Tính nhất quán giao dịch dữ liệu} (\ref{nfr:16}) và các mối quan hệ dữ liệu phức tạp giữa các Domain (Quiz, Course, Student).
		\item Độ phức tạp quản lý giao dịch phân tán (Saga) của MSA sẽ làm tăng chi phí và thời gian phát triển đáng kể.
		\item Mặc dù yêu cầu về tải có đỉnh điểm (mùa thi), quy mô hiện tại phù hợp hơn với kiến trúc ít phức tạp hơn.
	\end{enumerate}
	\item \textbf{Decision (Quyết định):}
	\begin{enumerate}
		\item Áp dụng \textbf{Service-Based Architecture (SBA)}, phân chia hệ thống thành các dịch vụ lớn (User/Access, Course Modification, Quiz/Assessment, Learning/Interaction, Reporting/Config).
		\item Sử dụng \textbf{API Gateway} để định tuyến, chứng thực và cung cấp lớp caching thống nhất (\ref{nfr:20}).
		\item Sử dụng \textbf{Service Registry} để quản lý vị trí động và trạng thái của các dịch vụ, hỗ trợ \textbf{Elasticity} (\ref{nfr:23}, \ref{nfr:24}) và \textbf{Availability} (\ref{nfr:10}, \ref{nfr:11}).
	\end{enumerate}
	\item \textbf{Consequences (Hệ quả):}
	\begin{enumerate}
		\item \textbf{Cải thiện Data Consistency (\ref{nfr:15}, \ref{nfr:16}):} Dễ dàng đạt được thông qua giao dịch cục bộ hoặc quản lý CSDL đơn giản hơn.
		\item \textbf{Giảm chi phí vận hành} và tăng tốc độ phát triển.
		\item Vẫn giữ được tính linh hoạt mở rộng nhờ các thành phần hỗ trợ (API Gateway, Service Registry).
	\end{enumerate}
\end{enumerate}