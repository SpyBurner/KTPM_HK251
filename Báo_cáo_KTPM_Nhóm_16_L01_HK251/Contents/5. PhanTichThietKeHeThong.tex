\section{Phân tích \& Thiết kế hệ thống}
\subsection{Entity Relationship Diagram (ERD)}
\subsubsection{Tổng quan}

\begin{figure}[H]
	\centering
	\includegraphics[width=0.8\linewidth]{"Images/5. PhanTichThietKeHeThong/ERD.png"}
	\caption{Entity Relationship Diagram}
	\label{fig:erd}
\end{figure}

\textbf{Tổng quan các Thực thể}\\
\indent Sơ đồ có tổng cộng 16 thực thể được chia thành 4 lĩnh vực (Domain): Quiz (Bài kiểm tra), Course Modification (Điều chỉnh Khóa học), Course Learning (Học tập Khóa học), và System (Hệ thống).

Course Modification, Quiz và Course Learning chia hệ thống thành 2 mảng tách rời nhưng tương trợ lẫn nhau: chỉnh sửa nội dung khóa học và sử dụng khóa học.

\begin{itemize}

	\item \textbf{Domain: Quiz (Bài kiểm tra)}\\
	Bao gồm các thực thể phục vụ định nghĩa các bài quiz, và ghi lại tương tác của Student trên các bài quiz này.
	\begin{itemize}
		\item \textbf{Quiz}: Đại diện cho một bài kiểm tra hoặc bài tập tổng thể.
		\item \textbf{Question}: Đại diện cho một câu hỏi cụ thể trong Quiz.
		\item \textbf{Answer}: Đại diện cho một lựa chọn trả lời khả dụng cho một Question.
		\item \textbf{Quiz Attempt}: Đại diện cho một lần làm Quiz cụ thể của một Student.
	\end{itemize}
	
	\item \textbf{Domain: Course Modification (Chỉnh sửa Khóa học)}\\
	Bao gồm các thực thể phục vụ định nghĩa các khóa học, phân chia một khóa học thành các đơn vị nhỏ, dễ truy xuất, tham chiếu.
	\begin{itemize}
		\item \textbf{Course}: Đại diện cho một khóa học hoàn chỉnh.
		\item \textbf{Chapter}: Đại diện cho một chương (bài học lớn) trong Course.
		\item \textbf{Topic}: Đại diện cho một chủ đề nhỏ hơn nằm trong Chapter.
		\item \textbf{Section}: Đại diện cho một phần nội dung, chứa các Quiz hoặc Topic.
	\end{itemize}
	
	\item \textbf{Domain: Course Learning (Học Khóa học)}\\
	Bao gồm các thực thể phục vụ các tương tác lên khóa học và quiz.
	\begin{itemize}
		\item \textbf{Forum message}: Đại diện cho một bài viết/bình luận trong diễn đàn.
	\end{itemize}
	
	\item \textbf{Domain: System (Hệ thống)}\\
	Bao gồm các thực thể phục vụ việc định danh người dùng, phân quyền, điều khiển và vận hành cho Admin.
	\begin{itemize}
		\item \textbf{User}: Đại diện cho người dùng chung của hệ thống (thực thể cha).
		\item \textbf{Admin / Dean / Lecturer / TA}: Đại diện cho người dùng có vai trò quản trị/giảng dạy, là thực thể con của User.
		\item \textbf{Student}: Đại diện cho người dùng có vai trò là học viên, là thực thể con của User.
		\item \textbf{File}: Đại diện cho các tài liệu được Upload lên hệ thống.
		\item \textbf{System config}: Đại diện cho các cài đặt cấu hình của toàn bộ hệ thống.
		\item \textbf{Privilege}: Đại diện cho các quyền hạn cụ thể của User.
		\item \textbf{Log}: Đại diện cho các bản ghi nhật ký hoạt động của User.
	\end{itemize}
\end{itemize}

\textbf{Các mối quan hệ giữa các thực thể}\\
\begin{longtable}{|p{1cm} |p{2cm} | p{1.5cm} | p{2cm} | p{6cm} |}
	\hline
	\textbf{Domain} & \textbf{Thực thể 1} & \textbf{Tên Mối Quan hệ} & \textbf{Thực thể 2} & \textbf{Kiểu Mối Quan hệ (Cardinality)} \\
	\hline
	
	Quiz & Quiz & Contain & Question & 1:N - Một Quiz có nhiều Question \\
	\hline
	Quiz & Question & Contain & Answer & 1:N - Một Question có nhiều Answer \\
	\hline
	Quiz & Quiz Attempt & Of & Quiz  & N:1 - Nhiều QuizAttempt thuộc về một Quiz \\
	\hline
	Quiz & Student & Take & Quiz Attempt & 1:N - Một Student thực hiện nhiều bài Quiz \\
	\hline
	Quiz & Admin / Dean / Lecturer / TA & Change & Quiz & 1:N - Một người dùng có quyền hạn có thể chỉnh sửa nhiều bài quiz \\
	\hline
	
	Course Mod. & Course & Contain & Chapter & 1:N - Một Course có nhiều Chapter \\
	\hline
	Course Mod. & Chapter & Contain & Section & 1:N - Một Chapter có nhiều Section \\
	\hline
	Course Mod. & Chapter & Contain & Topic & 1:N - Một Chapter có nhiều Topic \\
	\hline
	Course Mod. & Section & Contain & File  & 1:N - Một Section có nhiều file (tài liệu, video,...) \\
	\hline
	Course Mod. & Section & Contain & Quiz  & 1:N - Một Section có nhiều Quiz \\
	\hline
	Course Mod. & Admin / Dean / Lecturer / TA & Change & Course  & 1:N - Một người dùng có quyền hạn chỉnh sửa nhiều Course\\
	\hline
	Course Mod. & Admin / Dean / Lecturer / TA & Change & Chapter  & 1:N - Một người dùng có quyền hạn chỉnh sửa nhiều Chapter\\
	\hline
	Course Mod. & Admin / Dean / Lecturer / TA & Change & Section & 1:N - Một người dùng có quyền hạn chỉnh sửa nhiều Section\\
	\hline
	
	Course Learn. & Student & Finish & Chapter & N:N - Nhiều Student hoàn thành nhiều Chapter \\
	\hline
	Course Learn. & Student & Enroll & Course & N:N - Nhiều Student đăng ký nhiều Course \\
	\hline
	Course Learn. & User & Post & Forum message & 1:N - Một User đăng tải nhiều Forum message \\
	\hline
	Course Learn. & User & Post & Forum message & 1:N - Một User đăng tải nhiều Forum message \\
	\hline
	Course Learn. & Course & Contain & Forum message & 1:N - Một Course có nhiều Forum message \\
	\hline
	
	System & User & Upload & File & 1:1 - Một User đăng tải một File (ảnh đại diện) \\
	\hline
	System & User & Has & Privilege & N:N - Nhiều User có nhiều Privilege \\
	\hline
	System & Privilege & Grant access to & Course & N:1 - Nhiều Privilege cho phép truy cập một Course (2 dạng chỉ xem hoặc chỉnh sửa)\\
	\hline
	System & Log & Track & User & N:1 - Nhiều Log theo dõi một User - để truy vấn trách nhiệm khi có sửa đổi \\
	\hline
	System & Admin / Dean / Lecturer / TA & Change & System config & N:N - Nhiều người dùng có quyền hạn thay đổi nhiều System config \\
	\hline
	System & System config & Exclude & Course & 1:N - Một System config chứa thông tin về bỏ qua nhiều Course (hỗ trợ AI Service - use-case UC-SM-01)\\
	\hline
\end{longtable}

\textit{Mối Quan hệ Kế thừa (Inheritance/Specialization)}: Mối quan hệ này được biểu thị bằng ký hiệu vòng tròn \textbf{D} (Disjoint), nghĩa là người dùng Student sẽ không thể có được các quyền hạn ngang hàng với Admin, Lecturer, TA:

\textit{Mối Quan hệ bậc 3}: Mối quan hệ Contain N-N-N giữa Quiz Attempt, Question và Answer có vai trò lưu giữ câu trả lời của Student cho mỗi câu hỏi tương ứng trong một lần làm quiz.

\subsubsection{Mapping}
\begin{figure}[H]
	\centering
	\includegraphics[width=0.8\linewidth]{"Images/5. PhanTichThietKeHeThong/ERD(1).png"}
	\caption{Relational Mapping Diagram}
	\label{fig:mapping}
\end{figure}
Triển khai từ ERD ở phần trước, sơ đồ mapping bổ sung thêm các trường dữ liệu cho các bảng để hoàn thiện thiết kế cấu trúc dữ liệu.

\textbf{Một số chi tiết đáng lưu ý}:
\begin{itemize}
	\item Bảng AI Log chỉ dùng để lưu trữ các thông tin hoạt động của AI service, không có quan hệ với bảng nào khác.
	\item Tất cả các bảng đều có các trường createDateTime - thời điểm record đó được tạo ra, và isDeleted - hỗ trợ tính năng soft-delete
	\item Privilege được chia làm 2 loại (type): ACCESS - chỉ truy cập và đọc nội dung, dành cho Student, EDIT - truy cập và chỉnh sửa nội dung, dành cho Lecturer, TA.
\end{itemize}



\subsection{Kiến trúc hệ thống}
\subsubsection{Đặc tính kiến trúc (Architecture characteristics)}
Ứng dụng phương pháp \textbf{Architecture Katas} lên các yêu cầu hệ thống ở bảng \ref{tab:user_stories} và \ref{tab:nfr_list_ref}, nhóm đề xuất ra 3 đặc tính chính của hệ thống:

\begin{enumerate}
	\item \textbf{Configurability}, vì:
	
	\begin{itemize}
		\item \textbf{Quản lý hệ thống:} Cho phép thay đổi cách thức vận hành của hệ thống mà không cần can thiệp hoặc chỉnh sửa trực tiếp vào mã nguồn. Đặc tính này bao gồm việc \textit{Đối với mô hình Trí tuệ nhân tạo (AI)} thì có khả năng thiết lập \textit{chu kỳ tự động bảo trì}. Tương ứng với: \ref{nfr:01}, \ref{nfr:02}.
		
		\item \textbf{Chỉnh sửa khóa học:} Cho phép cấu hình \textit{kiểu tài liệu} người dùng được phép đăng tải, cũng như chỉnh sửa \textit{layout (bố cục) giao diện khóa học} hiển thị cho người dùng. Tương ứng với: \ref{nfr:03}, \ref{nfr:04}.
		
		\item \textbf{Tạo báo cáo:} Người dùng có thể \textit{thay đổi layout} của báo cáo được tạo ra và \textit{thay đổi phạm vi dữ liệu} được sử dụng để lập báo cáo. Tương ứng với: \ref{nfr:05}, \ref{nfr:06}.
		
		\item \textbf{Tạo câu đố bằng AI:} Cho phép cấu hình các tham số tạo câu hỏi như \textit{số lượng câu}, \textit{độ khó} mong muốn, và \textit{phân bổ độ khó} giữa các câu hỏi. Tương ứng với: \ref{nfr:07}.
		
		\item \textbf{Hỏi gợi ý:} Có thể thiết lập \textit{tần suất} hiển thị gợi ý và \textit{mức độ chi tiết} của gợi ý. Tương ứng với: \ref{nfr:08}.
		
		\item \textbf{Đề xuất chủ đề liên quan:} Cho phép người dùng \textit{thêm} các chủ đề quan tâm hoặc \textit{bỏ} các chủ đề không quan tâm ra khỏi danh sách đề xuất.Tương ứng với: \ref{nfr:09}.
	\end{itemize}
	
	\item \textbf{Availability}: Đặc tính này tập trung vào việc đảm bảo hệ thống luôn có thể được truy cập và sử dụng, đặc biệt là trong các tình huống \textit{thời gian sinh viên truy cập không ổn định} hoặc có sự tập trung người dùng lớn. Việc mất khả năng truy cập hệ thống sẽ ảnh hưởng đến chất lượng học tập của sinh viên - lớp người dùng luôn phải cân đối thời gian cho mọi việc, cũng như ảnh hưởng tiến độ xây dựng bài học của các giảng viên, trợ giảng,... Tương ứng với: \ref{nfr:10}, \ref{nfr:11}, \ref{nfr:12}.
	
	\item \textbf{Data consistency + integrity}, vì:
	\begin{itemize}
		\item Yêu cầu hệ thống phải \textit{đảm bảo tính đúng đắn} của các tài liệu bài giảng được đăng tải. Tương ứng với: \ref{nfr:13}.
		\item Yêu cầu \textit{dữ liệu đăng tải phải được giữ an toàn} khỏi các truy cập trái phép hoặc mất mát. Tương ứng với: \ref{nfr:14}.
		\item Đảm bảo \textit{mọi dữ liệu hiển thị phải đồng bộ} và nhất quán giữa tất cả các người dùng đang sử dụng hệ thống. Tương ứng với: \ref{nfr:15}.
		\item Các thao tác quan trọng phải đảm bảo tính toàn vẹn giao dịch (transactional integrity) và lịch sử chỉnh sửa nội dung khóa học phải được ghi lại để phục vụ truy vết. Tương ứng với: \ref{nfr:16}, \ref{nfr:17}
	\end{itemize}
\end{enumerate}

\textbf{Các đặc tính phụ liên quan}\\

Các đặc tính sau đây mô tả các yêu cầu hiệu năng và khả năng mở rộng của hệ thống:

\begin{itemize}
	\item \textbf{Performance (Hiệu năng):} Yêu cầu \textit{tối ưu hóa hiệu năng} của các mô hình AI để đảm bảo phản hồi nhanh chóng. Tương ứng với: \ref{nfr:18}, \ref{nfr:19}, \ref{nfr:20}.
	\item \textbf{Concurrency (Đồng thời):} Đảm bảo hệ thống có khả năng xử lý và phục vụ cho \textit{nhiều người dùng cùng lúc} mà không gặp lỗi hoặc giảm chất lượng dịch vụ. Tương ứng với: \ref{nfr:21}, \ref{nfr:22}.
	\item \textbf{Elasticity (Đàn hồi/Co giãn):} Đảm bảo hệ thống có khả năng mở rộng hoặc thu hẹp tài nguyên để đối phó với \textit{thời gian sử dụng không đều}, đặc biệt là khi tải trọng hệ thống \textit{tập trung cao vào mùa thi}. Tương ứng với: \ref{nfr:23}, \ref{nfr:24}, \ref{nfr:25}.
\end{itemize}
