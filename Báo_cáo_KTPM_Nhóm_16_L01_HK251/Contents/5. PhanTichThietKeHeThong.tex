\section{Phân tích \& Thiết kế hệ thống}
\subsection{Entity Relationship Diagram (ERD)}
\subsubsection{Tổng quan}

\begin{figure}[H]
	\centering
	\includegraphics[width=0.8\linewidth]{"Images/5. PhanTichThietKeHeThong/ERD.png"}
	\caption{Entity Relationship Diagram}
	\label{fig:erd}
\end{figure}

\textbf{Tổng quan các Thực thể}\\
\indent Sơ đồ có tổng cộng 16 thực thể được chia thành 4 lĩnh vực (Domain): Quiz (Bài kiểm tra), Course Modification (Điều chỉnh Khóa học), Course Learning (Học tập Khóa học), và System (Hệ thống).

Course Modification, Quiz và Course Learning chia hệ thống thành 2 mảng tách rời nhưng tương trợ lẫn nhau: chỉnh sửa nội dung khóa học và sử dụng khóa học.

\begin{itemize}

	\item \textbf{Domain: Quiz (Bài kiểm tra)}\\
	Bao gồm các thực thể phục vụ định nghĩa các bài quiz, và ghi lại tương tác của Student trên các bài quiz này.
	\begin{itemize}
		\item \textbf{Quiz}: Đại diện cho một bài kiểm tra hoặc bài tập tổng thể.
		\item \textbf{Question}: Đại diện cho một câu hỏi cụ thể trong Quiz.
		\item \textbf{Answer}: Đại diện cho một lựa chọn trả lời khả dụng cho một Question.
		\item \textbf{Quiz Attempt}: Đại diện cho một lần làm Quiz cụ thể của một Student.
	\end{itemize}
	
	\item \textbf{Domain: Course Modification (Chỉnh sửa Khóa học)}\\
	Bao gồm các thực thể phục vụ định nghĩa các khóa học, phân chia một khóa học thành các đơn vị nhỏ, dễ truy xuất, tham chiếu.
	\begin{itemize}
		\item \textbf{Course}: Đại diện cho một khóa học hoàn chỉnh.
		\item \textbf{Chapter}: Đại diện cho một chương (bài học lớn) trong Course.
		\item \textbf{Topic}: Đại diện cho một chủ đề nhỏ hơn nằm trong Chapter.
		\item \textbf{Section}: Đại diện cho một phần nội dung, chứa các Quiz hoặc Topic.
	\end{itemize}
	
	\item \textbf{Domain: Course Learning (Học Khóa học)}\\
	Bao gồm các thực thể phục vụ các tương tác lên khóa học và quiz.
	\begin{itemize}
		\item \textbf{Forum message}: Đại diện cho một bài viết/bình luận trong diễn đàn.
	\end{itemize}
	
	\item \textbf{Domain: System (Hệ thống)}\\
	Bao gồm các thực thể phục vụ việc định danh người dùng, phân quyền, điều khiển và vận hành cho Admin.
	\begin{itemize}
		\item \textbf{User}: Đại diện cho người dùng chung của hệ thống (thực thể cha).
		\item \textbf{Admin / Dean / Lecturer / TA}: Đại diện cho người dùng có vai trò quản trị/giảng dạy, là thực thể con của User.
		\item \textbf{Student}: Đại diện cho người dùng có vai trò là học viên, là thực thể con của User.
		\item \textbf{File}: Đại diện cho các tài liệu được Upload lên hệ thống.
		\item \textbf{System config}: Đại diện cho các cài đặt cấu hình của toàn bộ hệ thống.
		\item \textbf{Privilege}: Đại diện cho các quyền hạn cụ thể của User.
		\item \textbf{Log}: Đại diện cho các bản ghi nhật ký hoạt động của User.
	\end{itemize}
\end{itemize}

\textbf{Các mối quan hệ giữa các thực thể}\\
\begin{longtable}{|p{1cm} |p{2cm} | p{1.5cm} | p{2cm} | p{6cm} |}
	\hline
	\textbf{Domain} & \textbf{Thực thể 1} & \textbf{Tên Mối Quan hệ} & \textbf{Thực thể 2} & \textbf{Kiểu Mối Quan hệ (Cardinality)} \\
	\hline
	
	Quiz & Quiz & Contain & Question & 1:N - Một Quiz có nhiều Question \\
	\hline
	Quiz & Question & Contain & Answer & 1:N - Một Question có nhiều Answer \\
	\hline
	Quiz & Quiz Attempt & Of & Quiz  & N:1 - Nhiều QuizAttempt thuộc về một Quiz \\
	\hline
	Quiz & Student & Take & Quiz Attempt & 1:N - Một Student thực hiện nhiều bài Quiz \\
	\hline
	Quiz & Admin / Dean / Lecturer / TA & Change & Quiz & 1:N - Một người dùng có quyền hạn có thể chỉnh sửa nhiều bài quiz \\
	\hline
	
	Course Mod. & Course & Contain & Chapter & 1:N - Một Course có nhiều Chapter \\
	\hline
	Course Mod. & Chapter & Contain & Section & 1:N - Một Chapter có nhiều Section \\
	\hline
	Course Mod. & Chapter & Contain & Topic & 1:N - Một Chapter có nhiều Topic \\
	\hline
	Course Mod. & Section & Contain & File  & 1:N - Một Section có nhiều file (tài liệu, video,...) \\
	\hline
	Course Mod. & Section & Contain & Quiz  & 1:N - Một Section có nhiều Quiz \\
	\hline
	Course Mod. & Admin / Dean / Lecturer / TA & Change & Course  & 1:N - Một người dùng có quyền hạn chỉnh sửa nhiều Course\\
	\hline
	Course Mod. & Admin / Dean / Lecturer / TA & Change & Chapter  & 1:N - Một người dùng có quyền hạn chỉnh sửa nhiều Chapter\\
	\hline
	Course Mod. & Admin / Dean / Lecturer / TA & Change & Section & 1:N - Một người dùng có quyền hạn chỉnh sửa nhiều Section\\
	\hline
	
	Course Learn. & Student & Finish & Chapter & N:N - Nhiều Student hoàn thành nhiều Chapter \\
	\hline
	Course Learn. & Student & Enroll & Course & N:N - Nhiều Student đăng ký nhiều Course \\
	\hline
	Course Learn. & User & Post & Forum message & 1:N - Một User đăng tải nhiều Forum message \\
	\hline
	Course Learn. & User & Post & Forum message & 1:N - Một User đăng tải nhiều Forum message \\
	\hline
	Course Learn. & Course & Contain & Forum message & 1:N - Một Course có nhiều Forum message \\
	\hline
	
	System & User & Upload & File & 1:1 - Một User đăng tải một File (ảnh đại diện) \\
	\hline
	System & User & Has & Privilege & N:N - Nhiều User có nhiều Privilege \\
	\hline
	System & Privilege & Grant access to & Course & N:1 - Nhiều Privilege cho phép truy cập một Course (2 dạng chỉ xem hoặc chỉnh sửa)\\
	\hline
	System & Log & Track & User & N:1 - Nhiều Log theo dõi một User - để truy vấn trách nhiệm khi có sửa đổi \\
	\hline
	System & Admin / Dean / Lecturer / TA & Change & System config & N:N - Nhiều người dùng có quyền hạn thay đổi nhiều System config \\
	\hline
	System & System config & Exclude & Course & 1:N - Một System config chứa thông tin về bỏ qua nhiều Course (hỗ trợ AI Service - use-case UC-SM-01)\\
	\hline
\end{longtable}

\textit{Mối Quan hệ Kế thừa (Inheritance/Specialization)}: Mối quan hệ này được biểu thị bằng ký hiệu vòng tròn \textbf{D} (Disjoint), nghĩa là người dùng Student sẽ không thể có được các quyền hạn ngang hàng với Admin, Lecturer, TA:

\textit{Mối Quan hệ bậc 3}: Mối quan hệ Contain N-N-N giữa Quiz Attempt, Question và Answer có vai trò lưu giữ câu trả lời của Student cho mỗi câu hỏi tương ứng trong một lần làm quiz.

\subsubsection{Mapping}
\begin{figure}[H]
	\centering
	\includegraphics[width=0.8\linewidth]{"Images/5. PhanTichThietKeHeThong/ERD(1).png"}
	\caption{Relational Mapping Diagram}
	\label{fig:mapping}
\end{figure}
Triển khai từ ERD ở phần trước, sơ đồ mapping bổ sung thêm các trường dữ liệu cho các bảng để hoàn thiện thiết kế cấu trúc dữ liệu.

\textbf{Một số chi tiết đáng lưu ý}:
\begin{itemize}
	\item Bảng AI Log chỉ dùng để lưu trữ các thông tin hoạt động của AI service, không có quan hệ với bảng nào khác.
	\item Tất cả các bảng đều có các trường createDateTime - thời điểm record đó được tạo ra, và isDeleted - hỗ trợ tính năng soft-delete
	\item Privilege được chia làm 2 loại (type): ACCESS - chỉ truy cập và đọc nội dung, dành cho Student, EDIT - truy cập và chỉnh sửa nội dung, dành cho Lecturer, TA.
\end{itemize}



\subsection{Kiến trúc hệ thống}