\section{Giới thiệu}
\subsection{Đặt vấn đề}
Trong môi trường giáo dục truyền thống, việc giảng dạy chủ yếu diễn ra theo phương pháp đồng bộ, nơi giáo viên truyền đạt kiến thức chung cho cả lớp. Điều này tạo ra một số thách thức lớn:

\textbf{Giới hạn trong học tập cá nhân hóa}\\
\indent Việc giảng dạy truyền thống khiến giáo viên gặp khó khăn trong việc đáp ứng nhu cầu, tốc độ và phong cách học tập đa dạng của từng học viên.
\begin{itemize}
	\item Tốc độ và nhu cầu: Mỗi học viên có tốc độ học khác nhau, nhưng thời gian và nguồn lực hạn chế khiến việc cung cấp trải nghiệm học tập cá nhân hóa, phản hồi tức thì và hỗ trợ thích ứng theo thời gian thực trở thành một thách thức lớn.
	\item Thiếu hiệu quả: Điều này dẫn đến sự thiếu hiệu quả trong việc phát huy tối đa tiềm năng của mỗi người học.
\end{itemize}

\textbf{Sự thống trị của phương pháp giảng dạy thụ động}\\
\indent Các phương pháp giảng dạy truyền thống, nơi thông tin chảy một chiều từ người hướng dẫn đến học viên, thường dựa trên các hình thức thụ động.
\begin{itemize}
	\item Tỷ lệ lưu giữ kiến thức thấp: Theo nguyên lý của Tam giác Học tập (Learning Pyramid)\footnote{https://www.educationcorner.com/the-learning-pyramid/}, các phương pháp thụ động như Listen hay Read có tỷ lệ lưu giữ trung bình rất thấp (chỉ 5\% - 10\%).
	\begin{figure}[H]
		\centering
		\includegraphics[width=0.7\linewidth]{"Images/1. GioiThieu/Learning triangle.png"}
		\caption{Tam giác học tập}
		\label{fig:learning-tri}
	\end{figure}
	\item Thiếu kỹ năng mềm: Các phương pháp này không tạo điều kiện cho học viên tham gia vào các hoạt động học tập chủ động như giải quyết vấn đề, làm việc nhóm, hoặc giao tiếp , vốn là những kỹ năng mềm thiết yếu mà các chương trình kỹ thuật cần phải nâng cao.
\end{itemize}	

\textbf{Hạn chế trong đánh giá và theo dõi tiến độ học tập}\\
\indent Hệ thống giáo dục truyền thống chủ yếu dựa vào các bài kiểm tra định kỳ như giữa kỳ và cuối kỳ, dẫn đến việc đánh giá không phản ánh đầy đủ quá trình học tập của học viên.
\begin{itemize}
	\item Thiếu dữ liệu thời gian thực: Giáo viên không thể theo dõi tiến độ của từng học viên theo từng buổi học, dẫn đến khó phát hiện sớm những lỗ hổng kiến thức.
	\item Quá trình bị gián đoạn: Học viên chỉ nhận được phản hồi sau khi hoàn thành bài kiểm tra, nên khó điều chỉnh chiến lược học tập ngay lập tức.
	\item Không khuyến khích sự phát triển liên tục: Khi đánh giá chỉ tập trung vào điểm số, việc hình thành năng lực lâu dài và tư duy phản biện bị xem nhẹ.
\end{itemize}

\textbf{Thiếu đa dạng trong nội dung học tập và phương tiện giảng dạy}\\
\indent Phần lớn nội dung trong mô hình truyền thống được thiết kế theo chuẩn chương trình chung, không phân tầng theo trình độ và không linh hoạt theo nhu cầu cá nhân.
\begin{itemize}
	\item Ít mức độ truy cập: Học viên không có cơ hội lựa chọn nội dung phù hợp với khả năng—ví dụ: nâng cao, bổ sung cơ bản hoặc nội dung mở rộng.
	\item Thiếu phương tiện tương tác: Việc phụ thuộc vào sách giáo khoa và bài giảng trên lớp hạn chế việc tiếp cận tài liệu multimedia hoặc công cụ mô phỏng trực quan.
	\item Cản trở mô hình học tập tự định hướng: Học viên khó tiếp cận tài liệu ở mọi lúc, mọi nơi, dẫn đến sự bị động trong việc xây dựng lộ trình học riêng.
\end{itemize}

\subsection{Mô tả ý tưởng}
Để giải quyết những hạn chế của giáo dục truyền thống và tối đa hóa hiệu quả học tập, Hệ thống Gia sư Thông minh (ITS) đã được nghiên cứu và phát triển. ITS là một ứng dụng phần mềm mạnh mẽ sử dụng các kỹ thuật Trí tuệ Nhân tạo (AI) để mô phỏng và cung cấp hướng dẫn cá nhân hóa.
ITS tự điều chỉnh linh hoạt theo tốc độ, khả năng, điểm mạnh, điểm yếu và sở thích của từng người học thông qua các tính năng tiên tiến, tập trung vào học tập chủ động:
\begin{itemize}
	\item Tăng cường thực hành: ITS cung cấp bài tập thực hành thiết kế theo thời gian thực, kèm theo phản hồi thích ứng để đảm bảo học viên liên tục áp dụng kiến thức và phát triển kỹ năng giải quyết vấn đề.
	\item Hỗ trợ kỹ năng mềm: ITS tích hợp các module hỗ trợ thảo luận nhóm, giúp học viên luyện tập giao tiếp và hợp tác.
	\item Tự đánh giá và giải thích: Để đạt mức độ lưu giữ kiến thức cao nhất, ITS yêu cầu học viên giải thích các khái niệm thiết kế cho gia sư AI, buộc họ phải trở thành "người dạy", từ đó củng cố khả năng tự điều chỉnh (Self-Regulation) và hiểu sâu.
	\item Lộ trình học tập tùy chỉnh: AI của hệ thống phân tích chi tiết hoạt động học tập của học viên để tự động điều chỉnh lộ trình, gợi ý các tài liệu (như video, bài đọc) và chủ đề tiếp theo có độ khó phù hợp, thay thế cho chương trình giảng dạy chung cứng nhắc.
	\item Đánh giá liên tục: Thay vì chỉ dựa vào kiểm tra định kỳ, ITS cung cấp các bài kiểm tra ngắn sau mỗi nội dung của chương. Hệ thống AI có khả năng tự động tạo ra các câu hỏi đánh giá mới, đa dạng, phân tầng theo độ khó và tập trung vào các lỗ hổng kiến thức cụ thể của từng học viên.
	\item Phát hiện sớm lỗ hổng kiến thức: Mọi tương tác của học viên đều được ghi lại và phân tích chi tiết. Điều này giúp phát hiện sớm các lĩnh vực mà học viên đang gặp khó khăn, cho phép can thiệp và hỗ trợ kịp thời, đảm bảo không bỏ sót bất kỳ lỗ hổng kiến thức nào.
\end{itemize}

Nhờ những tính năng này, ITS cung cấp phản hồi, gợi ý và lộ trình học tập tùy chỉnh, mang lại hiệu quả tương đương với mô hình gia sư một kèm một, giúp người học phát huy tối đa tiềm năng của mình.